%-------------------------
% Resume in Latex
% Author : Jake Gutierrez
% Based off of: https://github.com/sb2nov/resume
% License : MIT
%------------------------

\documentclass[letterpaper,11pt]{article}

\usepackage{bibentry}
\usepackage{latexsym}
\usepackage[empty]{fullpage}
\usepackage{titlesec}
\usepackage{marvosym}
\usepackage[usenames,dvipsnames]{color}
\usepackage{verbatim}
\usepackage{enumitem}
\usepackage[hidelinks]{hyperref}
\usepackage{fancyhdr}
\usepackage[english]{babel}
\usepackage{tabularx}
\usepackage{url}

\input{glyphtounicode}

%----------FONT OPTIONS----------
% sans-serif
% \usepackage[sfdefault]{FiraSans}
% \usepackage[sfdefault]{roboto}
% \usepackage[sfdefault]{noto-sans}
% \usepackage[default]{sourcesanspro}

% serif
% \usepackage{CormorantGaramond}
% \usepackage{charter}


\pagestyle{fancy}
\fancyhf{} % clear all header and footer fields
\fancyfoot{}
\renewcommand{\headrulewidth}{0pt}
\renewcommand{\footrulewidth}{0pt}

% Adjust margins
\addtolength{\oddsidemargin}{-0.5in}
\addtolength{\evensidemargin}{-0.5in}
\addtolength{\textwidth}{1in}
\addtolength{\topmargin}{-.5in}
\addtolength{\textheight}{1.0in}

\urlstyle{same}

\raggedbottom
\raggedright
\setlength{\tabcolsep}{0in}

% Sections formatting
\titleformat{\section}{
  \vspace{-4pt}\scshape\raggedright\large
}{}{0em}{}[\color{black}\titlerule \vspace{-5pt}]

% Ensure that generate pdf is machine readable/ATS parsable
\pdfgentounicode=1

%-------------------------
% Custom commands
\newcommand{\resumeItem}[1]{
  \item\small{
    {#1 \vspace{-2pt}}
  }
}

\newcommand{\resumeSubheading}[4]{
  \vspace{-2pt}\item
    \begin{tabular*}{0.97\textwidth}[t]{l@{\extracolsep{\fill}}r}
      \textbf{#1} & #2 \\
      \textit{\small#3} & \textit{\small #4} \\
    \end{tabular*}\vspace{-7pt}
}

\newcommand{\resumeSubSubheading}[2]{
    \item
    \begin{tabular*}{0.97\textwidth}{l@{\extracolsep{\fill}}r}
      \textit{\small#1} & \textit{\small #2} \\
    \end{tabular*}\vspace{-7pt}
}

\newcommand{\resumeProjectHeading}[2]{
    \item
    \begin{tabular*}{0.97\textwidth}{l@{\extracolsep{\fill}}r}
      \small#1 & #2 \\
    \end{tabular*}\vspace{-7pt}
}

\newcommand{\resumeSubItem}[1]{\resumeItem{#1}\vspace{-4pt}}

\renewcommand\labelitemii{$\vcenter{\hbox{\tiny$\bullet$}}$}

\newcommand{\resumeSubHeadingListStart}{\begin{itemize}[leftmargin=0.15in, label={}]}
\newcommand{\resumeSubHeadingListEnd}{\end{itemize}}
\newcommand{\resumeItemListStart}{\begin{itemize}}
\newcommand{\resumeItemListEnd}{\end{itemize}\vspace{-5pt}}

%-------------------------------------------
%%%%%%  RESUME STARTS HERE  %%%%%%%%%%%%%%%%%%%%%%%%%%%%


\begin{document}

%----------HEADING----------
% \begin{tabular*}{\textwidth}{l@{\extracolsep{\fill}}r}
%   \textbf{\href{http://sourabhbajaj.com/}{\Large Sourabh Bajaj}} & Email : \href{mailto:sourabh@sourabhbajaj.com}{sourabh@sourabhbajaj.com}\\
%   \href{http://sourabhbajaj.com/}{http://www.sourabhbajaj.com} & Mobile : +1-123-456-7890 \\
% \end{tabular*}

\begin{center}
    \textbf{\Huge \scshape Luan da Fonseca Santos} \\ \vspace{1pt}
    \small +55 (11) 962035039 $|$ \href{luanfonseca93@gmail.com}{{luanfonseca93@gmail.com}}
    %\href{https://luanfs.github.io/}{luanfs.github.io/}
    %$|$    
    %\href{https://www.linkedin.com/in/luan-santos-b59a95168/}{{linkedin.com/in/luan-santos}}
\end{center}

%-----------Summary-----------
\section{Summary}
I have a PhD degree in Applied Mathematics from the University of São Paulo. 
My primary research focus is on the development of numerical methods for geophysical fluid dynamics, with a particular emphasis on numerical weather/climate modeling, specifically the dynamical core development.
I have experience in finite-volume/difference methods and locally refined grid generation.

%-----------EDUCATION-----------
\section{Education}
  \resumeSubHeadingListStart
    \resumeSubheading
{Institute of Mathematics and Statistics, University of São Paulo - USP}{São Paulo, SP, Brazil}
{Ph.D. Applied Mathematics}{March 2020 - May 2024}  

\begin{itemize}
	\item Thesis title: {Analysis of finite-volume advection
		schemes on cubed-sphere grids and an
		accurate alternative for divergent winds.} Advisor: Dr. Pedro Peixoto.
	\item \textbf{2023/09}: Research visit to the AOS - Princeton University (hosts: Dr. Joseph Mouallem and Dr. Lucas Harris).
\end{itemize}
      	
    \resumeSubheading
      {Institute of Mathematics and Statistics, University of São Paulo - USP}{São Paulo, SP, Brazil}
      {M. Sc. Applied Mathematics}{March 2018 - March 2020}
      \begin{itemize}
      	\item Dissertation title: \href{https://doi.org/10.11606/D.45.2020.tde-07052020-154350}{Analysis of mimetic finite volume schemes on classical and moist shallow water models considering topography based local refinement in spherical Voronoi grids.} Advisor: Dr. Pedro Peixoto.
      \end{itemize}
    \resumeSubheading
      {Institute of Mathematics and Statistics, University of São Paulo - USP}{São Paulo, SP, Brazil}
{B. Sc. Applied Mathematics (GPA: 9.3/10)}{2014 - 2017}
\begin{itemize}
	\item Project title: {Local refinement and interpolation in spherical icosahedral grids.} Advisor: Dr. Pedro Peixoto.
	\item Honorable mention for outstanding performance in the Applied Mathematics B.Sc. program. 
\end{itemize}
  \resumeSubHeadingListEnd


%-----------EXPERIENCE-----------
\section{Experience}
  \resumeSubHeadingListStart

    \resumeSubheading
      {Teaching Assistant}{2017 -- 2021}
      {University of São Paulo}{São Paulo, SP, Brazil} 
      \resumeItemListStart
        \resumeItem{Grad courses:}   
        \begin{itemize}[label=\textendash]
        \item $1^{\textrm{st}}$ sem/2019, $1^{\textrm{st}}$ sem/2020 and $1^{\textrm{st}}$ sem/2021 - MAP5729 - Introduction to Numerical Analysis (Institute of Mathematics and Statistics).
        \end{itemize} 
        \resumeItem{Undergrad courses:}   
        \begin{itemize}[label=\textendash]
    	\item $2^{\textrm{nd}}$ sem/2019 - MAP2320 - Numerical methods in differential equations II (Institute of Mathematics and Statistics).
    	\item {$2^{\textrm{nd}}$ sem/2018 - MAP0214 - Numerical Calculus with Applications to Physics (Institute of Astronomy, Geophysics and Atmospheric Sciences).}        
    	\item{$1^{\textrm{st}}$ sem/2017 - MAC0427 - Non-linear Programming (Institute of Mathematics and Statistics).} 
        \end{itemize}             

      \resumeItemListEnd
  \resumeSubHeadingListEnd

\section{Publication list}
\begin{itemize}
	\item Luan F Santos and Pedro S Peixoto (2021).  {\bf Topography based local spherical Voronoi grid refinement on classical and moist shallow-water finite volume models}, Geosci. Model Dev. Discuss., \href{https://doi.org/10.5194/gmd-14-6919-2021
	}{https://doi.org/10.5194/gmd-14-6919-2021}.
\end{itemize}


%-----------Talks-----------
\section{Talks}
\begin{itemize}
	\item \textbf{2023}: {Enhancing accuracy of FV3 finite-volume operators} at \textit{FV3 group meeting, 
	GFDL/NOAA, Princeton, USA}.
	\item \textbf{2021}: {Topography based local spherical Voronoi grid refinement on classical and moist shallow-water finite volume models} at \textit{PDEs on the sphere 2021, Offenbach, Germany (online)}.
	\item \textbf{2019}: Poster Presentation, \href{https://collaboration.cmc.ec.gc.ca/science/pdes-2019/pdfs/poster-Luan-Santos.pdf}{Topography based local refinement in spherical Voronoi grids} at \textit{PDEs on the sphere 2019, Montréal, Québec, Canada}.
\end{itemize}


%-----------Participation-----------
\section{Participation at events}
\begin{itemize}
	\item \textbf{2021}: Participation in the
	ESCAPE2/Fondazione Alessandro Volta  Summer school program -
	\textit{Towards exascale computing
		for numerical weather prediction, Lake Como School of Advanced Studies (online)}.
	\item \textbf{2019}: Participation in the \textit{Winter School in Atmospheric Numerical Modeling at CPTEC (Center for Weather Forecasting and Climate Studies), Cachoeira Paulista, SP, Brazil}.
\end{itemize}


%
%---------- GRANTS-----------
\section{Grants}
\begin{itemize}
	\item Doctoral degree scholarship  - São Paulo Research Foundation (FAPESP),  grant 20/10280-4, 2020-2024.		
	\item Master's degree scholarship   - São Paulo Research Foundation (FAPESP),  grant 17/25191-4, 2018-2020.
	\item Undergraduate research funding  - São Paulo Research Foundation (FAPESP), grant 17/11542-0, 2017.
\end{itemize}

 
%
%---------- Links-----------
\section{Links}
\begin{itemize}
	\item Personal webpage:  \href{https://luanfs.github.io/}{https://luanfs.github.io/}
	\item Google scholar: \href{https://scholar.google.com/citations?user=lHQVgqgAAAAJ&hl}{https://scholar.google.com/citations?user=lHQVgqgAAAAJ\&hl}
	\item ORCID:  \href{https://orcid.org/0000-0001-9084-6170}{https://orcid.org/0000-0001-9084-6170}
\end{itemize}

 
%
%---------- Technical Skills -----------
\section{Technical Skills}
\begin{itemize}
	\item Programming languages: Fortran, Python (NumPy, SciPy, Matplotlib, Cartopy), C, and Matlab.
	\item Experience with parallel programming using OpenMP and MPI.
	\item General software and tools: Linux environment, Bash scripts, Git, remote servers, SSH, Tmux, Vim, \LaTeX.
\end{itemize}


%---------- Additional Information -----------
\section{Additional Information}
\begin{itemize}
	\item Date of birth: August 07, 1993.
	\item Citizenship: Brazilian.
	\item Gender: Male.
	\item Marital status: Married. %\item Children: None.
	\item Languages: Portuguese (native) and English (advanced).
\end{itemize}




%-------------------------------------------
\end{document}

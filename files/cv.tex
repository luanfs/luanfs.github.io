%-------------------------
% Resume in Latex
% Author : Jake Gutierrez
% Based off of: https://github.com/sb2nov/resume
% License : MIT
%------------------------

\documentclass[letterpaper,11pt]{article}

\usepackage{bibentry}
\usepackage{latexsym}
\usepackage[empty]{fullpage}
\usepackage{titlesec}
\usepackage{marvosym}
\usepackage[usenames,dvipsnames]{color}
\usepackage{verbatim}
\usepackage{enumitem}
\usepackage[hidelinks]{hyperref}
\usepackage{fancyhdr}
\usepackage[english]{babel}
\usepackage{tabularx}
\usepackage{url}

\input{glyphtounicode}

%----------FONT OPTIONS----------
% sans-serif
% \usepackage[sfdefault]{FiraSans}
% \usepackage[sfdefault]{roboto}
% \usepackage[sfdefault]{noto-sans}
% \usepackage[default]{sourcesanspro}

% serif
% \usepackage{CormorantGaramond}
% \usepackage{charter}


\pagestyle{fancy}
\fancyhf{} % clear all header and footer fields
\fancyfoot{}
\renewcommand{\headrulewidth}{0pt}
\renewcommand{\footrulewidth}{0pt}

% Adjust margins
\addtolength{\oddsidemargin}{-0.5in}
\addtolength{\evensidemargin}{-0.5in}
\addtolength{\textwidth}{1in}
\addtolength{\topmargin}{-.5in}
\addtolength{\textheight}{1.0in}

\urlstyle{same}

\raggedbottom
\raggedright
\setlength{\tabcolsep}{0in}

% Sections formatting
\titleformat{\section}{
  \vspace{-4pt}\scshape\raggedright\large
}{}{0em}{}[\color{black}\titlerule \vspace{-5pt}]

% Ensure that generate pdf is machine readable/ATS parsable
\pdfgentounicode=1

%-------------------------
% Custom commands
\newcommand{\resumeItem}[1]{
  \item\small{
    {#1 \vspace{-2pt}}
  }
}

\newcommand{\resumeSubheading}[4]{
  \vspace{-2pt}\item
    \begin{tabular*}{0.97\textwidth}[t]{l@{\extracolsep{\fill}}r}
      \textbf{#1} & #2 \\
      \textit{\small#3} & \textit{\small #4} \\
    \end{tabular*}\vspace{-7pt}
}

\newcommand{\resumeSubSubheading}[2]{
    \item
    \begin{tabular*}{0.97\textwidth}{l@{\extracolsep{\fill}}r}
      \textit{\small#1} & \textit{\small #2} \\
    \end{tabular*}\vspace{-7pt}
}

\newcommand{\resumeProjectHeading}[2]{
    \item
    \begin{tabular*}{0.97\textwidth}{l@{\extracolsep{\fill}}r}
      \small#1 & #2 \\
    \end{tabular*}\vspace{-7pt}
}

\newcommand{\resumeSubItem}[1]{\resumeItem{#1}\vspace{-4pt}}

\renewcommand\labelitemii{$\vcenter{\hbox{\tiny$\bullet$}}$}

\newcommand{\resumeSubHeadingListStart}{\begin{itemize}[leftmargin=0.15in, label={}]}
\newcommand{\resumeSubHeadingListEnd}{\end{itemize}}
\newcommand{\resumeItemListStart}{\begin{itemize}}
\newcommand{\resumeItemListEnd}{\end{itemize}\vspace{-5pt}}

%-------------------------------------------
%%%%%%  RESUME STARTS HERE  %%%%%%%%%%%%%%%%%%%%%%%%%%%%


\begin{document}

%----------HEADING----------
% \begin{tabular*}{\textwidth}{l@{\extracolsep{\fill}}r}
%   \textbf{\href{http://sourabhbajaj.com/}{\Large Sourabh Bajaj}} & Email : \href{mailto:sourabh@sourabhbajaj.com}{sourabh@sourabhbajaj.com}\\
%   \href{http://sourabhbajaj.com/}{http://www.sourabhbajaj.com} & Mobile : +1-123-456-7890 \\
% \end{tabular*}

\begin{center}
    \textbf{\Huge \scshape Luan da Fonseca Santos} \\ \vspace{1pt}
    \href{https://luanfs.github.io}{https://luanfs.github.io} $|$
    ls9640@princeton.edu
    $|$  
    {luanfsantos14@gmail.com} 
    %$|$    
    %\href{https://www.linkedin.com/in/luan-santos-b59a95168/}{{linkedin.com/in/luan-santos}}
\end{center}

%-----------Summary-----------
\section{Summary}
I am an applied mathematician and currently a Postdoctoral Researcher Associate at Princeton University working with the FV3 team at the Geophysical Fluid Dynamics Laboratory (GFDL).
My primary research focuses on the development, analysis, and implementation of numerical methods for global atmospheric modeling.
%More specifically, I work with finite-volume and finite-difference methods.

%-----------Research interests-----------
%\section{Research interests}
%\begin{itemize}
%	\item Finite-volume/difference methods, spherical grids, grid imprinting.
%\end{itemize}

%-----------EDUCATION-----------
\section{Education}
\resumeSubHeadingListStart
%\resumeSubheading
%{Princeton University}{Princeton, NJ, USA}
%{Postdoctoral Researcher Associate}{July 2024 - Current}  
%$\quad$
\resumeSubheading
{University of São Paulo }{São Paulo, SP, Brazil}
{Ph.D. in Applied Mathematics}{March 2020 - May 2024}  
\begin{itemize}
	\item Thesis title: \href{https://www.teses.usp.br/teses/disponiveis/45/45132/tde-29052024-125153/en.php}{Analysis of finite-volume advection
	schemes on cubed-sphere grids and an
	accurate alternative for divergent winds.}
%	\item With financial support from São Paulo Research Foundation (FAPESP),  grant \href{https://bv.fapesp.br/en/bolsas/194920/analysis-and-development-of-finite-volume-methods-for-the-new-generation-of-cubed-sphere-dynamical-c/}{20/10280-4} and CAPES.
\end{itemize}
\resumeSubSubheading{M. Sc. in Applied Mathematics}{March 2018 - March 2020}
\begin{itemize}
	\item Dissertation title: \href{https://www.teses.usp.br/teses/disponiveis/45/45132/tde-07052020-154350/en.php}{Analysis of mimetic finite volume schemes on classical and moist shallow water models considering topography based local refinement in spherical Voronoi grids.}
%	\item With financial support from São Paulo Research Foundation (FAPESP), grant \href{https://bv.fapesp.br/en/bolsas/176667/analysis-of-finite-volume-schemes-for-spherical-geodesic-grids-with-local-refinement/}{17/25191-4}.
\end{itemize}

\resumeSubSubheading{B. Sc. in Applied Mathematics (GPA: 9.3/10)}{February 2014 - December 2017}
\begin{itemize}
	\item Undergraduate thesis: {Local refinement and interpolation in spherical icosahedral grids.}
	\item Honorable mention for outstanding performance in the Applied Mathematics B.Sc. program. 
\end{itemize}
\resumeSubHeadingListEnd

%-----------Research Visit-----------
\section{Research visits}
\begin{itemize}
	\item September 2023 -
	Princeton University - Atmospheric \& Oceanic Sciences Program.
\end{itemize}



%-----------EXPERIENCE-----------
\section{Experience}
\resumeSubHeadingListStart
\resumeSubheading
{Cooperative Institute for Modeling the Earth System,  Princeton University}{Princeton, NJ, USA}{Postdoctoral Researcher Associate}{July 2024 - Present}  
\resumeItemListStart
\resumeItem{Implementing and evaluating specific enhancements to FV3's numerical algorithms to advance the accuracy and efficiency of FV3-based weather and climate models. Supervisors: Dr. Lucas Harris and Dr. Joseph Mouallem.}
%\resumeItem{ Working at the U.S. National Oceanic and Atmospheric Adminstration’s Geophysical Fluid Dynamics Laboratory (NOAA-GFDL) with the FV3 team.} 
\resumeItemListEnd

\resumeSubheading
{Institute of Mathematics and Statistics, University of São Paulo}{São Paulo, SP, Brazil}{Graduate Researcher Student}
{March 2018 - May 2024}  
\resumeItemListStart
\resumeItem{Numerical methods for PDEs, finite-volume/difference schemes for geophysical fluid dynamics on a sphere, locally refined Voronoi grids, cubed-sphere grids. With financial support from São Paulo Research Foundation (FAPESP),  grant numbers	\href{https://bv.fapesp.br/en/bolsas/176667/analysis-of-finite-volume-schemes-for-spherical-geodesic-grids-with-local-refinement/}{17/25191-4},
\href{https://bv.fapesp.br/en/bolsas/194920/analysis-and-development-of-finite-volume-methods-for-the-new-generation-of-cubed-sphere-dynamical-c/}{20/10280-4}.
Supervisor: Dr. Pedro Peixoto.}
%\resumeItem{Developed topography-based locally refined spherical Voronoi grids for South America and implemented a finite-volume moist shallow-water model to evaluate their effectiveness.}
%\resumeItem{Developed and implemented an alternative enhanced transport scheme for the NOAA-GFDL FV3 model, achieving minimal additional computational overhead.}
%\resumeItem{Participated in a two-week research visit to the Atmospheric and Oceanic Sciences Program at Princeton University.}
\resumeItemListEnd

\resumeSubSubheading
{Undergraduate Researcher Student}{July 2017- December 2017}
\resumeItemListStart
\resumeItem{Worked on the implementation of algorithms for generating topography-based, locally refined spherical Voronoi grids on the sphere. Funded by FAPESP, grant number \href{https://bv.fapesp.br/en/bolsas/172369/local-refinement-and-interpolation-in-spherical-icosahedral-grids/}{17/11542-0}. Supervisor: Dr. Pedro Peixoto.}
\resumeItemListEnd

\resumeSubSubheading
{Teaching Assistant}{}
\resumeItemListStart
\resumeItem{Grad courses: $1^{\textrm{st}}$ sem/2019, $1^{\textrm{st}}$ sem/2020 and $1^{\textrm{st}}$ sem/2021 - MAP5729 - Introduction to Numerical Analysis.}
\resumeItem{Undergrad courses:}   
\begin{itemize}[label=\textendash]
	\item $2^{\textrm{nd}}$ sem/2019 - MAP2320 - Numerical methods for PDEs.
	\item {$2^{\textrm{nd}}$ sem/2018 - MAP0214 - Numerical Calculus with Applications to Physics.}        
	\item{$1^{\textrm{st}}$ sem/2017 - MAC0427 - Non-linear Optimization.} 
\end{itemize}       
\resumeItemListEnd

\resumeSubheading
{Inst. of Astronomy, Geophysics and Atmospheric Sci., University of São Paulo }{São Paulo, SP, Brazil}
{Part-time Computer Lab Monitor}{January 2015 - July 2016}
\resumeItemListStart
\resumeItem{Ensured smooth operation of computer lab hardware and software, providing technical support to students and resolving any issues promptly.}        
\resumeItemListEnd
\resumeSubHeadingListEnd

%-----------Publication list-----------
\section{Publication list}
\begin{itemize}
	\item Luan F. Santos and Pedro S. Peixoto (2024).
	{ Analysis of finite-volume transport schemes on cubed-sphere grids and an accurate scheme for divergent winds}, 
	{\bf In review}, Preprint: \href{http://dx.doi.org/10.2139/ssrn.4866660}{http://dx.doi.org/10.2139/ssrn.4866660}.
	\item Luan F. Santos and Pedro S. Peixoto (2021).
	{Topography based local spherical Voronoi grid refinement on classical and moist shallow-water finite volume models}, Geosci. Model Dev. Discuss., \href{https://doi.org/10.5194/gmd-14-6919-2021
	}{https://doi.org/10.5194/gmd-14-6919-2021}.
\end{itemize}




%-----------Events-----------
\section{Presentations and participation at events}
\begin{itemize}
	\item \textbf{2023}: Talk at the \textit{FV3 group meeting}, 
	GFDL/NOAA, Princeton, USA: \textit{Enhancing accuracy of FV3 finite-volume operators}.
	\item \textbf{2021}: Participation in the
    ESCAPE2/Fondazione Alessandro Volta  Summer school program -
    \textit{Towards exascale computing
	for numerical weather prediction, Lake Como School of Advanced Studies (online)}.
	\item \textbf{2021}: Talk at \textit{PDEs on the sphere 2021}, Offenbach, Germany (online): \textit{Topography based local spherical Voronoi grid refinement on classical and moist shallow-water finite volume models.} 
	\item \textbf{2019}: Participation in the \textit{Winter School in Atmospheric Numerical Modeling at CPTEC (Center for Weather Forecasting and Climate Studies), Cachoeira Paulista, SP, Brazil}.
	\item \textbf{2019}: Poster Presentation at \textit{PDEs on the sphere 2019}, Montréal, Québec, Canada:
	\textit{\href{https://collaboration.cmc.ec.gc.ca/science/pdes-2019/pdfs/poster-Luan-Santos.pdf}{Topography based local refinement in spherical Voronoi grids.}}
\end{itemize}

 

%---------- Referee Activities -----------
\section{Referee Activities}
\begin{itemize}
	\item Meteorological Applications (2024).
\end{itemize}
%
%---------- Technical Skills -----------
\section{Technical Skills}
\begin{itemize}
	\item Programming languages: Fortran, Python, C, and Matlab.
	\item Experience with parallel programming using OpenMP and MPI.
	\item General software and tools: Linux environment, Bash scripts, Git, remote servers, Vim, \LaTeX.
\end{itemize}


%
%---------- Links-----------
\section{Links}
\begin{itemize}
	\item Personal webpage:  \href{https://luanfs.github.io/}{https://luanfs.github.io/}
	\item Google scholar: \href{https://scholar.google.com/citations?user=D-uXvM0AAAAJ&hl=en}{https://scholar.google.com/citations?user=D-uXvM0AAAAJ\&hl=en}
	%\item ORCID:  %\href{https://orcid.org/0000-0001-9084-6170}{https://orcid.org/0000-0001-9084-6170}
\end{itemize}


%---------- Additional Information -----------
\section{Additional Information}
\begin{itemize}
	\item Citizenship: Brazilian.
	\item Languages: Portuguese (native) and English (advanced).
\end{itemize}




%-------------------------------------------
\end{document}
